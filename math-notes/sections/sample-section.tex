\documentclass[../AAL-Notes.tex]{subfiles}

\begin{document}

The contents of this section are in the corresponding file (\texttt{sample-section.tex}) in the \texttt{sections} folder. I'm using the \texttt{subfile} package to organize the document.  

All images go into the \texttt{images} folder. Here's a sample:

\begin{center}
	\includegraphics[scale = 0.45]{images/barney}
\end{center}


\subsection{Sample Math Environments}
You can edit the appearance of these in \texttt{main.tex}. The package used is \texttt{tcolorbox} whose documentation can be found easily online.

\textbf{Note:} The syntax for calling these environments is slightly different from the usual \texttt{amsthm} package, so please take a look at \texttt{sample-section.tex} before using it.  

\begin{definition}{Barney \& Friends}{}
Barney \& Friends is an American children's television series aimed at children from ages 1 to 8, created by Sheryl Leach and produced by HIT Entertainment. It premiered on PBS on April 6, 1992. The series features the title character Barney, a purple anthropomorphic Tyrannosaurus rex who conveys educational messages through songs and small dance routines with a friendly, optimistic attitude.
\end{definition}

The above definition was taken from. Wikipedia.  

\begin{proposition}{Barney is Purple}{}
Barney is a purple dinosaur. 
\end{proposition}
\begin{proof}
Left as an exercise to the reader. 
\end{proof}
 
 
\begin{lemma}{}{}
Dinosaurs are extinct. 
\end{lemma}

\begin{corollary}{}{}
Sample corollary, I've run out of Barney jokes. 
\end{corollary}

\begin{example}{}{}
\begin{enumerate}
  \item Example 1. 
  \item Example 2. 
\end{enumerate}
\end{example}


\begin{theorem}{}{}
Hope this template is of help to you! 
\end{theorem}
 
\end{document}
